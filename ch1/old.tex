\hypertarget{ch1:introduction}{%
\section{Introduction}\label{ch1:introduction}}

We developed and deployed an improved deep sea hydraulic engine
(Fig.~\ref{ch1:fig:teaser}) for \glspl{FEA}. The engine is designed to deploy on
small deep sea platforms such as science class \glspl{ROV}. The design can be
adapted and optimized for different use cases.  The prototype successfully tested with multiple
actuators \cite{becker2020mechanically,teeple2020multi} 
which were operated at depth on multiple
deployments with work-class \glspl{ROV}. During these trials, the prototype engine successfully
demonstrated biological sampling capabilities.

\begin{figure}
\hypertarget{ch1:fig:teaser}{%
\centering
\includegraphics[width=\textwidth,height=\textheight,keepaspectratio]{actuation.png}
\caption{Prototype deep sea hydraulic drive demonstrating actuation of a
\gls{FEA} manipulator \cite{teeple2020multi} in the unactuated position (top) and the
actuated position (bottom). The system fully actuates in under \SI{1}{\second}.}\label{ch1:fig:teaser}
}
\end{figure}

\hypertarget{ch1:importance}{%
\subparagraph{Importance}\label{ch1:importance}}

We focused on \glspl{FEA} over other soft robotic options due to their ease of
manufacturing, pressure tolerance,and low-cost 
\cite{yin2023rapid}\footnote{\textcite{yin2023rapid}: R. Shomberg is co-author (see appendix)}.
While \gls{FEA} manipulators have been shown to have applications in underwater
sampling \cite{gruber2022advances,mazzeo2022marine,galloway2016soft, subad2021soft, liu2020underwater, tessler2020ultra}, and
bio-mimicry \cite{marchese2014autonomous, chen2022neural, katzschmann2016cyclic, tan2021underwater}, the size and power
requirements of currently available deep sea drives limit deployments to
large platforms with large support vessels mitigating many of the
advantages to using soft robotics. The improved drive is more compact
and draws less power than previous engines enabling soft robotics to be
deployed on smaller platforms. Reducing platform sizes also reduces or
eliminates support vessel requirements significantly reducing operation
costs \cite{amon2022deep}. The size and power reduction also allows larger
platforms to more easily deploy a soft robotic system alongside other
equipment. Thus the improved drive enables a range of new applications
for underwater soft robotics.



\hypertarget{ch1:related-work}{%
\paragraph{Related Work}\label{ch1:related-work}}

\hypertarget{ch1:standard-practice}{%
\subparagraph{Standard Practice}\label{ch1:standard-practice}}

Soft robotics are primarily used in factory or lab settings and powered
pneumatically a centralized compressor which distributes air to as many
actuators as needed \cite{xavier2022soft, walker2020soft}. 
The compressor builds and stores a large volume of high pressure air. 
The \glspl{FEA} are generally actuated by feed-backwards control 
using solenoids and pressure limiting valves close to the actuator \cite{walker2020soft}. Actuation rates depend on the pressure differential and tubing diameters. 
Feed-forward volume control is rarely used due to the compressibility of air
and simplicity of measuring air pressure. 
Research on actuation rate optimization has focused on actuator design \cite{xavier2022soft} 
rather than drive improvements.

While pneumatic soft actuators report high power densities, the
centralized compressor is not usually accounted for in power density or
efficiency calculations \cite{wehner2014pneumatic}. Outside of laboratory settings, drive
characteristics become more important \cite{tolleymichael2014resilient}.



\hypertarget{ch1:state-of-the-art}{%
\subparagraph{State-of-the-Art}\label{ch1:state-of-the-art}}

\hypertarget{ch1:difficulty}{%
\subparagraph{Difficulty}\label{ch1:difficulty}}

Pneumatic power is not possible for deep sea operation 
due to high ambient pressures which compress gas \cite{gruber2022advances}. 
Pneumatic actuators have been successfully deployed underwater, 
but are limited to relatively shallow depths \cite{gruber2022advances}.
Hydraulic deep sea soft actuator drives have been demonstrated 
with water as the working fluid \cite{chen2021water}. 
The relative incompressibility of water allows for operation under high ambient pressures.  
However, hydraulic systems require greater power than pneumatic systems
especially to achieve high flow rates or rapid flow acceleration
\cite{xavier2022soft}.

Hydraulic experiments with \glspl{FEA} have generally swapped the working medium 
from pneumatic to hydraulic without much consideration of the potential advantages or optimizations. 
It is important to note that the \gls{FEA} internal pressure is a response to internal volume. Most research reports pressure because feedback control using pressure limiting
valves are readily available and accurate for pnuematics \cite{chen2021water}, 
and output volume for compressible flow has a non-linear relationship to input
volume. However, for hydraulic systems control can be achieved through
feed-forward control where output volume is equal to input
volume, and both pressure and actuation are functions of volume. Many
experimental hydraulic systems ignore this relationship and continue to
utilize pressure feedback control. \textcite{chen2021water} shows a potential advantage
of hydraulics and volume control by using a very small volume actuator
with an integrated drive.

\hypertarget{ch1:key-projects}{%
\subparagraph{Key Projects}\label{ch1:key-projects}}

Increased interest in soft robotics for deep-sea applications such as
biological sampling \cite{galloway2016soft,licht2016universal,licht2017stronger,phillips2018dexterous} and bio-inspired
robotics \cite{ahmed2022decade,li2023bioinspired} has created a growing need for improved drive
designs. However, only a handful of projects have deployed deep sea soft-robotic
engines. \textcite{galloway2016soft} demonstrated a dual cylinder engine using
the platform's hydraulic power to control a high-pressure oil-filled
cylinder mechanically tied to a low-pressure water water-filled
cylinder. The system was successfully demonstrated on a science class \gls{ROV}
\cite{licht2017stronger} but struggled to maneuver due to the large bulk of the system.
\textcite{phillips2018dexterous} demonstrated an open seawater pump which utilized an accumulator tank and valve manifold to release pressure. The
system offered multiple degrees of freedom and required just
\SI{50}{\watt}. However, the system was bulky and operated at low flow
rates due to the limited pressure differential and complex flow paths. \textcite{shen2020underwater} produced a similar open seawater pump which  utilized soft accumulators for storage. The system is much less bulky
than previous systems and was demonstrated on an observation class \gls{ROV}.

\hypertarget{ch1:dual-cylinder-pump}{%
\subparagraph{Dual Cylinder Pump}\label{ch1:dual-cylinder-pump}}



\hypertarget{ch1:open-sea-water-pump}{%
\subparagraph{Open Sea Water Pump}\label{ch1:open-sea-water-pump}}



\hypertarget{ch1:soft-open-sea-water-pump}{%
\subparagraph{Soft Open Sea Water Pump}\label{ch1:soft-open-sea-water-pump}}

\hypertarget{ch1:solution}{%
\paragraph{Solution}\label{ch1:solution}}

\hypertarget{ch1:proposed-solution}{%
\subparagraph{Proposed Solution}\label{ch1:proposed-solution}}

The developed hydraulic engine design (Fig.~\ref{ch1:fig:drawing}) uses a
single closed-volume of hydraulic fluid. The fluid required to operate the output
actuator is stored in a compressible soft bellows. An electronically
controlled servo compresses the bellows forcing fluid out of the engine
into the output actuator. The sealed engine housing is filled with oil
for pressure tolerance. The servo is a position controlled stepper servo
allowing for repeatable feed-forward volume control. Each hydraulic
engine provides a single \gls{DOF} for the output actuator which can consist
of multiple \glspl{FEA} on a single manifold. For additional \glspl{DOF}, additional
engines are required.

% The design uses a flexible bellows instead of a piston. Unlike
% a piston, the bellows is friction-less and does not rely on a sliding
% seal. However, the bellows does have a spring-force response to
% compression and is limited by a maximum ``squirm pressure''.
% Additionally, the bellows cannot produce significant negative pressure. The system operates on a single stroke which must transfer the
% full volume while overcoming back pressure from the actuator and dynamic
% effects.

\hypertarget{ch1:design-considerations}{%
\subparagraph{Design Considerations}\label{ch1:design-considerations}}

% The characteristics of the described system are dependent on servo selection and bellows
% design. These changes allow optimizations for specific output actuators
% and use cases. Bellows geometry determines the output volume which must
% be greater than the input volume of the output actuator. Internal
% bellows pressure is determined by the pressure-response of the output actuator's
% and system's hydraulic dynamic response. Rapid
% actuation produces a higher bellows pressure. Internal pressure is
% limited the squirm limit of the bellows and feed-force limit of the
% servo. The dynamic response is also dependent on the hydraulic tube
% dimensions and the compliance, or fluidic capacitance, of the bellows
% and output actuator.

% The pressure limits changes with bellows compression. Under rapid
% actuation, the bellows will reach much higher pressures than the output
% actuator. Maximizing the pressure limit for a given output volume
% optimizes the actuation rate and power density. The bellows pressure
% limits depend on competing design consideration. In general, for a given
% output volume, the feed-force limit is greater for a long skinny
% bellows, but the squirm limit is greater for a short fat bellows. We
% proposed a simplified design process which can be further optimized
% through iterative design.

\hypertarget{ch1:success-criteria}{%
\subparagraph{Success Criteria}\label{ch1:success-criteria}}


We created a design process for developing versions of the hydraulic
engine for use with different output actuators and use cases. The design
process incorporates the results of a mathematical model describing
engine characteristics and dynamics. We verified the model through lab
testing. Previous project have not reported actuation rate for deep sea
soft robotics \cite{li2023bioinspired}. Our modeling and testing establishes a baseline
for drive speed and power density The prototypes tested show higher
power density allowing for much more compact packages allowing them to
be deployed on smaller platforms.

\hypertarget{ch1:methods}{%
\section{Methods}\label{ch1:methods}}



\hypertarget{ch1:design-overview}{%
\paragraph{Design Overview}\label{ch1:design-overview}}

\begin{figure}
\hypertarget{ch1:fig:drawing}{%
\centering
\includegraphics[width=\textwidth,height=\textheight,keepaspectratio]{concept.png}
\caption{Concept drawing showing the linear drive hydraulic engine in
unactuated (A) and actuated (B) states. The engine uses a servo to
compress a soft bellows forcing fluid down the flexible tube into the
actuator. The system is contained in an oil-filled pressure-compensated
housing. The rear end-cap uses a flexible diaphragm to allow for the
internal volume change caused by operation.}\label{ch1:fig:drawing}
}
\end{figure}

The linear hydraulic drive created in this study (Fig.~\ref{ch1:fig:drawing}) consists of an
electric servo which compresses an flexible bellows pushing water into
the soft actuator through a tube. The electric servo is a position
controlled stepper servo allowing for precise feed forward control and
position holding. The system uses water as the working fluid which
converts the position of the electric servo to a precise input volume.
The flexible bellows compresses axially expelling fluid while also
allowing for radial expansion to absorb back pressure. Unlike a cylinder
drive, the bellows has no sliding seals which produce frictional
resistance to rapid actuation and are potential sources for leaking. The
design of the hydraulic tube has an effect on the dynamic response of
the system. The engine operates inside an oil-filled housing for pressure
compensation. A flexible diaphragm allows for the internal volume
changes. Control electronics are in a separate \US{1}{\atm} housing.

% TODO: move this somewhere better
Using system modeling, the actuation rate can be optimized for a
given electric servo while meeting the volume output requirements
through changes in bellows and tube parameters. Changes can be
implemented quickly through rapid manufacturing processes.

\hypertarget{ch1:results}{%
\section{Results}\label{ch1:results}}


\hypertarget{ch1:deep-sea-field-trials}{%
\subsection{Deep-Sea Field Trials}\label{ch1:deep-sea-field-trials}}

We successfully deployed a prototype of the hydraulic engine on multiple deep-sea \gls{ROV}
missions. During deployment, the engine was placed in a custom
pressure-tolerant oil-filled housing and used thin-walled round bellows
with three convolutions made from molded silicone. These field
deployments, which provided a proof of concept of the pressure compensation/linear drive scheme, occurred before the laboratory testing and modeling efforts. Lessons from the field tests were incorporated into the modeling efforts and proposed future work.


\hypertarget{ch1:prototype-design}{%
\paragraph{Prototype Design}\label{ch1:prototype-design}}

The linear drive and bellows are housed in an oil filled pressure
compensated cylinder (Fig.~\ref{ch1:fig:housing}). The housing consists of
two acrylic tubes with 3D printed end-caps and connecting fitting. The
front end-cap includes ports for filling the housing with oil and a port
to connect tubing going to a fluidic actuator. The rear end-cap includes
a standard wet-mateable underwater connector to the motor controller. A
flexible diaphragm sealed to the back of the rear end-cap equalizes the
internal pressure with ambient pressure and allows the internal volume
of the housing to change. The total internal volume changes whenever the
linear drive is actuated because liquid is propelled in and out from the
bellows. Due to the oil pressure compensation and lack of compressible
volumes inside the system, the oil-filled housing allows the engine to
theoretically operate at any depth

\begin{figure}
\hypertarget{ch1:fig:housing}{%
\centering
\includegraphics[width=\textwidth,height=\textheight,keepaspectratio]{render2real.png}
\caption{Render (left) and real (right) design}\label{ch1:fig:housing}
}
\end{figure}

\hypertarget{ch1:control-1}{%
\paragraph{Control}\label{ch1:control-1}}

The hydraulic engine is controlled using industrial automation
architecture, with a Festo CMMO12 controller\footnote{https://www.festo.com/gb/en/}
contained in a \US{1}{\atm}
electronics bottle. The linear drive is controlled by a Festo CMMO
controller which communicates with an external platform (ROV, manned
submersible, etc.) via a 100~Mbps Ethernet connection. The controller
accepts industry standard MODBUS\footnote{https://modbus.org/} \gls{TCP}
commands allowing near-real-time
network control of the linear drive with a minimum update rate of \SI{1}{\milli\second}.
Commands are sent to the CMMO using MODBUS \gls{TCP} protocols through a
static IP connection. This connection can be further integrated into a
larger network, with multiple CMMO's driving multiple actuator fuctions
simultaneously. Commands to the CMMO can also be sent by any device
capable of sending MODBUS \gls{TCP} commands. For direct control over the
device using a joystick or similar controller, a \gls{PLC} sends MODBUS \gls{TCP} commands. The \gls{PLC} used in the
presented design is a Unitronics UniStream \gls{PLC} with built-in touch
screen, chosen for its ease of programming. The \gls{PLC} can accept analog
inputs from a joystick or similar device, allowing for relatively simple
development for broader system architecture.

The CMMO itself fits into a \US{1}{\atm} cylindrical housing with
internal dimensions of \SI{113}{\mm} diameter and \SI{115}{\mm} length. The depth
rating of the motor controller housing determines the overall depth
rating of the system since all other components are pressure tolerant.
In the field test described in this paper, a \SI{6000}{\meter} rated housing
was chosen based on the needs of concurrent experiments which utilized
the same housing. The controller connects to the linear drive motor via
a harsh-environment wet-mateable underwater cable with four control
wires and two brake wires, and operates as a stepper motor controller to
operate the linear drive.


\hypertarget{ch1:deployment}{%
\paragraph{Deployment}\label{ch1:deployment}}

\begin{figure}
\hypertarget{ch1:fig:onrov}{%
\centering
\includegraphics[width=\textwidth,height=\textheight,keepaspectratio]{onrov.png}
\caption{Prototype drive deployed on deep sea platforms. Clockwise
starting in top left: \gls{ROV} Hercules on deck, \gls{ROV} SuBastion on deck, and
\gls{ROV} Hercules at depth with target of opportunity.}\label{ch1:fig:onrov}
}
\end{figure}

\begin{figure}
\hypertarget{ch1:fig:targets}{%
\centering
\includegraphics[width=\textwidth,height=\textheight,keepaspectratio]{targets.png}
\caption{Biological Targets-of-opportunity used to test the hydraulic
engine include a stalked tunicate (upper-left), Venus flytrap sea
anemone (upper-right) and an extremely fragile xenophyophore (bottom).
Targets were grasped and released without any apparent injury and
minimal disturbance.}\label{ch1:fig:targets}
}
\end{figure}

Our prototype system was deployed (Fig.~\ref{ch1:fig:onrov}) on multiple
platforms. First deployment was with the \gls{ROV} SuBastian during a deep-sea
expedition off the Northwest coast of Oahu, Hawaii in October 2019. The
drive controller was integrated into the \gls{ROV} network and operated via a
laptop over Ethernet. The pilots used the main \gls{ROV} grippers to position
the fluidic actuators (grippers) over targets. The system was tested
successfully at depths up to \SI{1500}{\meter}. The deployed drive utilized
the molded bellows design to operate a dual-finger pinch grasping array \cite{teeple2020multi}
and a high aspect ratio soft entanglement actuator array \cite{becker2020mechanically}. 
The fluidic actuators operated at approximately \US{15}{\psi} over ambient pressure. 
The team conducted tests at a variety of depths, including mid-water and
sea-floor, using encountered organisms as targets-of-opportunity
(Fig.~\ref{ch1:fig:targets}). As expected, the fluidic actuators gripped
organisms without injuring them because of the natural compliance of the
fluidic actuators. Notably, actuation of the soft grippers was
qualitatively much faster and repeatable than any previous system
deployed at depth. Successfully grasped targets included fly trap
anemones, glass sponges, sea stars, and a xenophyophore. The successful
grasping of a xenophyophore was particularly notable as no damage
to its extremely delicate shell was observed.

The prototype was also deployed for a second field test on \gls{ROV} Hercules
in October of 2020 utilizing the larger \gls{SLA} printed bellows to drive a
the same dual-finger pinch grasping array with minor modifications. Operators
used the system to collect delicate sea urchins. The system successfully
operated at a depth of approximately \SI{2000}{\meter} and reached a max depth of \SI{3870}{\meter} without damage.

\hypertarget{ch1:discussion}{%
\paragraph{Discussion}\label{ch1:field-discussion}}

The prototype deployed successfully as a proof of concept. However,
during deployment, we identified multiple pitfalls which are addressed
in the resulting design proposal. During field test we observed 
lower than expected pressure and actuation rate limitations.
The silicone bellows was far more compliant than the printed
versions. This resulted in significant expansion as well as a delayed
response at the output actuator. Additionally, the increased compliance
resulted in decreased axial stiffness causing off axis buckling. For
deployments, we printed a bellows guide to constrain bellows movement.
While somewhat effective, the guide also increased system friction. Even
with the guide in place, we observed that dynamic back pressure also limited
actuation rate. These lessons helped inform the modeling and design
proposal.


\hypertarget{ch1:system-model}{%
\subsection{System Model}\label{ch1:system-model}}

\hypertarget{ch1:design-fabrication-and-optimization}{%
\paragraph{Design, Fabrication, and
Optimization}\label{ch1:design-fabrication-and-optimization}}

We have developed a system model and design process to improve future system performance.
The model attempts to predict the limitations observed during field demonstrations.
The design process uses the model to predict parameters to improve actuation rate for a given output.
Due to the highly non-linear nature and variability of the modeling, 
the proposed design process does not represent a full optimization.
In this effort, modeling is performed using simulated data. 
We propose a set of lab tests to confirm model validity.

Engine design depends on the output actuator and operational
requirements. The following design process can be followed using the
input parameters of a given output actuator and linear servo. The design
process is iterative and may result in identifying a need for a different
servo.

\hypertarget{ch1:system-overview}{%
\paragraph{System Overview}\label{ch1:system-overview}}

\begin{figure}
\hypertarget{ch1:fig:model}{%
\centering
\includegraphics[width=\textwidth,height=\textheight,keepaspectratio]{system-model.png}
\caption{Hydraulic flow diagram depicting a drive consisting of a
bellows with connecting tube driving an actuator. The bellows depicted
is a thin-walled triangular bellows with \(N=3\) convolutions of height
\(h_0\) and wall thickness \(t\). The bellows has an inner \(D_i\) and
outer \(D_o\) diameters. As the bellows is compressed axially distance
\(x\), fluid is pushed through a tube of diameter \(d_t\) and length
\(l_t\) until the actuator pressure \(p_a\) equalizes with bellows
pressure \(p_b\).}\label{ch1:fig:model}
}
\end{figure}

Fig.~\ref{ch1:fig:model} shows the hydraulic flow diagram which results in
flow out of the bellows and into the actuator.

\hypertarget{ch1:system-elements}{%
\paragraph{System Elements}\label{ch1:system-elements}}

\hypertarget{ch1:output-actuator}{%
\paragraph{Output Actuator}\label{ch1:output-actuator}}

The system is meant to be tuned to the output actuator. Sets of
actuators on a manifold are treated as a single output actuator. \gls{FEA}
actuation can take the form of complex motion including combinations of
extension, bending, and/or twisting. While the actuation geometry
depends on design and material properties, the motion is a function
internal volume. For drive design, we ignore resulting actuator motion
and use internal volume change \(V_a\) as a proxy for complex motion. If
necessary, the resulting motion can be calculated later. The resulting
actuator pressure \(p_a\) response and the maximum volume change
\(\Delta V_\text{max}\) completely describes the output actuator for the
purposes of engine design.

We model the output actuator as a fluidic capacitance
(Eq.~\ref{ch1:eq:actuator-pressure}, ~\ref{ch1:eq:actuator-energy}).
The pressure response \(p_a\) and energy storage \(\mathcal{E}_a\) of the
output actuator are measurable functions of internal volume change \(\Delta V_a\). 
The hydraulic drive must supply the maximum fluid volume \(\Delta V_{a,\text{max}}\)
at sufficient pressure \(p_{a,\text{max}}\)
to fully operate the actuator. Volume changes are measured relative to
initial internal volume at ambient pressure.
For modeling, we define (Eq.~\ref{ch1:eq:actuator-capacitance}) a linear approximate for the
actuator response using the ideal actuator capacitance coefficient
\(C_a\). Every output actuator has a unique pressure response. While
useful for initial design, the linear approximation is not valid for
many available actuators.

\begin{equation}\protect\hypertarget{ch1:eq:actuator-pressure}{}{ 
p_a(\Delta V_a) 
\approx \frac{1}{C_f} \Delta V_a
}\label{ch1:eq:actuator-pressure}\end{equation}
\begin{equation}\protect\hypertarget{ch1:eq:actuator-energy}{}{ \mathcal{E}_a = \int_{0}^{p_a} p_a dV_a
\approx \frac{1}{2} C_a p_a^2 
}\label{ch1:eq:actuator-energy}\end{equation}
\begin{equation}\protect\hypertarget{ch1:eq:actuator-capacitance}{}{ C_a: = \frac{ \Delta V_{a,\text{max}} }{ p_{a,\text{max}} } 
}\label{ch1:eq:actuator-capacitance}\end{equation}

\hypertarget{ch1:output-actuators}{%
\paragraph{Output Actuators}\label{ch1:output-actuators}}

We used a single actuator with a measured pressure response curve. The
output is based on the design previously by the authors \cite{yin2023rapid}\footnote{\textcite{yin2023rapid}: R. Shomberg is co-author (see appendix)}.
We approximated a linear response based on the measured maximum volume and
pressure to get a single constant capacitance value
Tbl.~\ref{ch1:tbl:output}.

\hypertarget{ch1:tbl:output}{}
\begin{longtable}[]{@{}
  >{\raggedright\arraybackslash}p{(\columnwidth - 4\tabcolsep) * \real{0.4750}}
  >{\raggedleft\arraybackslash}p{(\columnwidth - 4\tabcolsep) * \real{0.0875}}
  >{\raggedright\arraybackslash}p{(\columnwidth - 4\tabcolsep) * \real{0.4375}}@{}}
\caption{\label{ch1:tbl:output}Output actuator parameters}\tabularnewline
\toprule\noalign{}
\begin{minipage}[b]{\linewidth}\raggedright
Parameter
\end{minipage} & \begin{minipage}[b]{\linewidth}\raggedleft
Value
\end{minipage} & \begin{minipage}[b]{\linewidth}\raggedright
Units
\end{minipage} \\
\midrule\noalign{}
\endfirsthead
\toprule\noalign{}
\begin{minipage}[b]{\linewidth}\raggedright
Parameter
\end{minipage} & \begin{minipage}[b]{\linewidth}\raggedleft
Value
\end{minipage} & \begin{minipage}[b]{\linewidth}\raggedright
Units
\end{minipage} \\
\midrule\noalign{}
\endhead
\bottomrule\noalign{}
\endlastfoot
Max volume \(\Delta V_{a,\text{max}}\) & 50 & \si{\milli\liter} \\
Max pressure \(p_{a,\text{max}}\) & 47 & \si{\kilo\pascal} \\
Capacitance \(C_a\) & 1.1 & \si{\milli\liter\per\kilo\pascal} \\
\end{longtable}


\hypertarget{ch1:linear-servo}{%
\paragraph{Linear Servo}\label{ch1:linear-servo}}

The chosen linear servo Tbl.~\ref{ch1:tbl:servo} is the Festo
EPCO-25-100-10P. The servo has rapid actuation with accurate feed
forward position control. We expect the feed force to be the limiting
factor any design.

\hypertarget{ch1:tbl:servo}{}
\begin{longtable}[]{@{}
  >{\raggedright\arraybackslash}p{(\columnwidth - 4\tabcolsep) * \real{0.4304}}
  >{\raggedleft\arraybackslash}p{(\columnwidth - 4\tabcolsep) * \real{0.0886}}
  >{\raggedright\arraybackslash}p{(\columnwidth - 4\tabcolsep) * \real{0.4810}}@{}}
\caption[\label{ch1:tbl:servo}Stepper-servo parameters Festo
EPCO-25-100-10P]{\label{ch1:tbl:servo}Stepper-servo parameters Festo
EPCO-25-100-10P\footnote{https://www.festo.com/us/en/p/electric-actuator-id\_EPCO/}}\tabularnewline
\toprule\noalign{}
\begin{minipage}[b]{\linewidth}\raggedright
Parameter
\end{minipage} & \begin{minipage}[b]{\linewidth}\raggedleft
Value
\end{minipage} & \begin{minipage}[b]{\linewidth}\raggedright
Units
\end{minipage} \\
\midrule\noalign{}
\endfirsthead
\toprule\noalign{}
\begin{minipage}[b]{\linewidth}\raggedright
Parameter
\end{minipage} & \begin{minipage}[b]{\linewidth}\raggedleft
Value
\end{minipage} & \begin{minipage}[b]{\linewidth}\raggedright
Units
\end{minipage} \\
\midrule\noalign{}
\endhead
\bottomrule\noalign{}
\endlastfoot
Feed Force \(F_\text{max}\) & 105 & \si{\newton} \\
Stroke \(x_\text{max}\) & 100 & \si{\milli\meter} \\
Feed Velocity \(v_\text{max}\) & 500 & \si{\milli\meter\per\second} \\
Feed Acceleration \(a_\text{max}\) & 10 &
\si{\milli\meter\per\second\squared} \\
\end{longtable}


\hypertarget{ch1:bellows}{%
\paragraph{Bellows}\label{ch1:bellows}}

The bellows is the primary design element of the engine. The bellows
internal volume \(V_b\) changes from a combination
(Eq.~\ref{ch1:eq:bellows-volume}) of axial compression \(\Delta V_{b,x}\)
and radial expansion \(\Delta V_{b,r}\).

\begin{equation}\protect\hypertarget{ch1:eq:bellows-volume}{}{ \Delta V_b = \Delta V_{b,x} + \Delta V_{b,r}   
}\label{ch1:eq:bellows-volume}\end{equation}

\hypertarget{ch1:bellows-compression}{%
\subparagraph{Bellows Compression}\label{ch1:bellows-compression}}

The axial compression is a function (Eq.~\ref{ch1:eq:compression-axial}) of
compression distance \(x\) which can be linear approximated using the
bellows effective cross-sectional area \(A_b\).

\begin{equation}\protect\hypertarget{ch1:eq:compression-axial}{}{ \Delta V_{b,x}(x) \approx -A_b x   
}\label{ch1:eq:compression-axial}\end{equation}
\[ \Delta V_{b,\text{max}} \approx -A_b x_\text{max}   
\]

\hypertarget{ch1:bellows-expansion}{%
\subparagraph{Bellows Expansion}\label{ch1:bellows-expansion}}

The radial expansion is a function (Eq.~\ref{ch1:eq:expansion-radial}) of
internal bellows pressure \(p_b\) and compression distance \(x\). The
bellows compliance that allows for radial expansion can be approximated
using an ideal fluidic capacitance coefficient \(C_b\). However, the
bellows capacitance depends on initial geometry and axial compression.
Radial expansion is an approximation for any non-axial expansion. It may
not be strictly radial in a traditional sense.

\begin{equation}\protect\hypertarget{ch1:eq:expansion-radial}{}{ \Delta V_{b,r}(p_b,x) \approx \frac{1}{C_b} p_b   
}\label{ch1:eq:expansion-radial}\end{equation}
\begin{equation}\protect\hypertarget{ch1:eq:bellows-energy}{}{ \mathcal{E}_b(p_b,x) = \int_{0}^{p_a} p_b dV_{b,r}(p_b,x)
\approx \frac{1}{2} C_b p_b^2 
}\label{ch1:eq:bellows-energy}\end{equation}

\hypertarget{ch1:volume-limit}{%
\paragraph{Volume Limit}\label{ch1:volume-limit}}

Maximum volume output \(\Delta V_{a,\text{max}}\) depends on maximum
volume change of the bellows \(\Delta V_{b,\text{max}}\) which depends
on the effective area \(A_b\) and stroke length \(x_\text{max}\). The
bellows retains internal volume even at full compression depending on
design geometries..

\[ \Delta V_{a,\text{max}} \le \Delta V_{b,\text{max}}   
= A_b x_\text{max}
\]

\hypertarget{ch1:pressure-limits}{%
\paragraph{Pressure Limits}\label{ch1:pressure-limits}}

The system operates under two opposing pressure limits. The feed force
limit is based on servo limitations and is greater for The squirm limit
is based on bellows limitations. Peak bellows pressure can be much
higher than actuator operating pressure depending on the speed of
actuation. Higher pressure limits allow for greater actuation rates.

\hypertarget{ch1:feed-force-limit}{%
\subparagraph{Feed Force Limit}\label{ch1:feed-force-limit}}

Axial compression force \(F\) on the bellows is a combination
(Eq.~\ref{ch1:eq:axial-force}) of forces from the bellows pressure force
\(F_b\) and spring force \(F_K\).

\begin{equation}\protect\hypertarget{ch1:eq:axial-force}{}{ F(x,p_b) = F_K(x) + F_b(p_b)   
}\label{ch1:eq:axial-force}\end{equation}

The bellow pressure force \(F_b\) is a function
(Eq.~\ref{ch1:eq:force-pressure}) of the bellows pressure \(p_b\), and can
be linearly approximated using the bellows effective cross-sectional
area \(A_b\).

\begin{equation}\protect\hypertarget{ch1:eq:force-pressure}{}{ F_b(p_b) \approx A_b p_b   
}\label{ch1:eq:force-pressure}\end{equation}

The spring force \(F_K\) is a function (Eq.~\ref{ch1:eq:force-spring}) of
axial compression \(x\). It can be be approximated using a stiffness
\(K\). However, the stiffness depends on geometry, material properties,
and axial compression. Spring energy \(\mathcal{E}_k\) is stored as a
function Eq.~\ref{ch1:eq:spring-energy} of axial compression.

\begin{equation}\protect\hypertarget{ch1:eq:force-spring}{}{ F_K(x) = x \cdot K(x)   
}\label{ch1:eq:force-spring}\end{equation}
\begin{equation}\protect\hypertarget{ch1:eq:spring-energy}{}{ \mathcal{E}_K(x) = \int_0^F F dx   
\approx \frac{1}{2} \frac{F_K(x)^2}{K(x)}   
}\label{ch1:eq:spring-energy}\end{equation}

The feed force is limited by the maximum servo feed force
\(F_\text{max}\) resulting in a bellows pressure limit
(Eq.~\ref{ch1:eq:feed-limit}).

\begin{equation}\protect\hypertarget{ch1:eq:feed-limit}{}{ 
p_b(x) \le \frac{ F_\text{max} - x \cdot K(x) }{ A_b }   
}\label{ch1:eq:feed-limit}\end{equation}

\hypertarget{ch1:squirm-pressure-limit}{%
\subparagraph{Squirm Pressure Limit}\label{ch1:squirm-pressure-limit}}

\begin{figure}
\hypertarget{ch1:fig:squirm}{%
\centering
\includegraphics[width=\textwidth,height=\textheight,keepaspectratio]{squirm-example.png}
\caption{Example of bellows squirm. Under high internal pressure, the
bellows exerts axial force on the constraining ends. The resulting load
causes axial instability causing the bellows to buckle off axes as
indicated by the overlayed lines.}\label{ch1:fig:squirm}
}
\end{figure}

Similar to an elastic columns \cite{hermann1997static}, large axial forces can cause a
bellows to which the bellows will buckle off axis
(Fig.~\ref{ch1:fig:squirm}). This failure mode is known as squirm. Bellows
squirm occurs suddenly above the critical pressure \(p_{cr}\) of the
bellows. The pressure limit depends (Eq.~\ref{ch1:eq:squirm-limit}) on the
stiffness \(K\), initial length \(L_0\), and compression distance \(x\)
\cite{hermann1997static}.

\begin{equation}\protect\hypertarget{ch1:eq:squirm-limit}{}{ 
p_{cr}(x) = \frac{2\pi}{L_0-x} \cdot K(x)   
}\label{ch1:eq:squirm-limit}\end{equation}


\hypertarget{ch1:prototype-bellows-design}{%
\paragraph{Prototype Bellows
Design}\label{ch1:prototype-bellows-design}}

\hypertarget{ch1:bellows-geometry}{%
\paragraph{Bellows Geometry}\label{ch1:bellows-geometry}}

\begin{figure}
\hypertarget{ch1:fig:bellows-params}{%
\centering
\includegraphics[width=\textwidth,height=\textheight,keepaspectratio]{bellows-params.png}
\caption{Diagram of basic geometry references showing geometric
parameters described in Tbl.~\ref{ch1:tbl:bellows-params} for thin-walled
triangular bellows. The dashed right triangle shows the relationship
between parameters. The bellows is corrugated thin-walled cylinder with
axial symmetry made of a flexible material.}\label{ch1:fig:bellows-params}
}
\end{figure}

Bellows can be manufactured in many different shapes. For lab testing,
we designed and fabricated a flexible thin-walled triangular bellows
(Fig.~\ref{ch1:fig:bellows-params}).


\hypertarget{ch1:tbl:bellows-params}{}
\begin{longtable}[]{@{}ll@{}}
\caption{\label{ch1:tbl:bellows-params}Interdependent geometric
(Fig.~\ref{ch1:fig:bellows-params}) parameters of a thin-walled triangular
bellows. Bolded parameters do not change with axial compression \(x\).
The bellows can be described completely by the first five independent
parameters.}\tabularnewline
\toprule\noalign{}
Parameter & Symbol \\
\midrule\noalign{}
\endfirsthead
\toprule\noalign{}
Parameter & Symbol \\
\midrule\noalign{}
\endhead
\bottomrule\noalign{}
\endlastfoot
\textbf{Inner Diameter} & \(D_i\) \\
Convolution Height & \(h\) \\
\textbf{Number of Convolutions} & \(N\) \\
\textbf{Material Thickness} & \(t\) \\
Convolution Angle & \(\theta\) \\
Overall Bellows Length & \(L\) \\
Outer Diameter & \(D_o\) \\
\textbf{Convolution Hypotenuse} & \(S\) \\
\end{longtable}

This geometry is easy to manufacture and can be defined using only a few
parameters (Tbl.~\ref{ch1:tbl:bellows-params}). These parameters determine
the bellow properties including volume and pressure limits. The geometry
changes with axial compression \(x\) from the initial state. Geometry
changes from radial expansion are assumed insignificant. Strain limiting
is used to prevent expansion of the inner diameter. Assuming bending
occurs evenly and only at the inner and outer convolution angles, we can
define the geometric parameters based on the initial parameters and
axial compression \(x\).

Each half-convolution forms a right triangle with fixed hypotenuse \(S\)
defining relationships between the remaining parameters. We assume the
hypotenuse and inner diameters remain fixed.

\[ 
S^2 = \left( \frac{h}{2} \right)^2 +   
    \left(\frac{D_i (\alpha-1)}{2} \right)^2   
\] \[ 
\tan{\frac{\theta}{2}} = \frac{h}{D_i (\alpha-1)}   
\]

These can be rearranged to become functions of axial compression \(x\)
and the initial convolution height \(h_0\).

\[ 
h(x) = h_0 - \frac{x}{N}   
\] \[ 
\alpha(x) = 1 + \sqrt{4S^2 - h(x)^2}   
\]

For a thin-walled triangular bellows the effective cross-sectional area
\(A_b\) is the inside area of the end-caps which for our design is based
off (Eq.~\ref{ch1:eq:bellows-area}) the inner diameter \(D_i\) of the
bellows.

\begin{equation}\protect\hypertarget{ch1:eq:bellows-area}{}{A_b = \frac{\pi}{4} D_i^2
}\label{ch1:eq:bellows-area}\end{equation}

The maximum compression distance \(x_\text{max}\) is limited
(Eq.~\ref{ch1:eq:max-compression}) by the number \(N\) of convolutions,
initial convolution height \(h_0\), and material thickness \(t\). At
maximum compression, the convolution walls touch. The remaining volume
depends on the compressed length and the inner area. The maximum bellows
volume change results (Eq.~\ref{ch1:eq:max-volume}) from the max compression
distance.

\begin{equation}\protect\hypertarget{ch1:eq:max-compression}{}{ 
x \le N(h_0 - 2t)   
}\label{ch1:eq:max-compression}\end{equation}
\begin{equation}\protect\hypertarget{ch1:eq:max-volume}{}{
\Delta V_{b,x} \le A_b  N(h_0 - 2t)   
}\label{ch1:eq:max-volume}\end{equation}

The bellows stiffness \(K\) depends (Eq.~\ref{ch1:eq:stiffness}) on
geometry, material properties (Tbl.~\ref{ch1:tbl:materials}), and
diameter-ratio \(\alpha\) which is a function (Eq.~\ref{ch1:eq:alpha}) of
compression distance \(x\) \cite{wang2020mechanoreception}.

\begin{equation}\protect\hypertarget{ch1:eq:stiffness}{}{ 
K(\alpha) = \frac{\pi E t^3}{3N(1-\nu^2)D_i^2} \left[
\frac{1}{\ln{\alpha} - (\alpha-1) + \frac{1}{2}(\alpha-1)^2}
\right]   
}\label{ch1:eq:stiffness}\end{equation}
\begin{equation}\protect\hypertarget{ch1:eq:alpha}{}{ 
\alpha(x) = D_o(x)/D_i   
}\label{ch1:eq:alpha}\end{equation}

Bellows capacitance \(C_b\) depends (Eq.~\ref{ch1:eq:bellows-capacitance})
on geometry and compression distance \(x\). This approximation is based
on the hoop strain for a thin-walled cylinder of equivalent average
diameter.

\begin{equation}\protect\hypertarget{ch1:eq:bellows-capacitance}{}{ C_b(x) \approx   
\frac{\pi}{16}  
\frac{\left( \frac{1}{2} (\alpha(x)+1) D_i \right)^3} {E t}   
\left( Nh_0-x \right)
}\label{ch1:eq:bellows-capacitance}\end{equation}

\hypertarget{ch1:bellows-fabrication}{%
\paragraph{Bellows Fabrication}\label{ch1:bellows-fabrication}}

Bellows are fabricated using \gls{FDM} printing with flexible \gls{TPU} material.
The bellows were printed vertically around their axis without supports.
Printer overhang was limited to \SI{45}{\degree}, so bellows were designed
with \SI{90}{\degree} convolutions. To ensure water-tightness even when
pressurized, the walls were printed with four (4) horizontal layers. When
measured, the thickness was slightly greater than expected. The
as-fabricated values (Tbl.~\ref{ch1:tbl:materials}) were used for prototype
design and modeling.


\hypertarget{ch1:tbl:materials}{}
\begin{longtable}[]{@{}lrl@{}}
\caption{\label{ch1:tbl:materials}Material properties for NinjaFlex
filament printed on a Prusa \gls{FDM} printer.}\tabularnewline
\toprule\noalign{}
Property & Value & Units \\
\midrule\noalign{}
\endfirsthead
\toprule\noalign{}
Property & Value & Units \\
\midrule\noalign{}
\endhead
\bottomrule\noalign{}
\endlastfoot
Elastic Modulus \(E\) & 12 & \si{\mega\pascal} \\
Poisson's ratio \(\nu\) & 0.26 & \\
Wall-thickness \(t\) & 1.62 & \si{\milli\meter} \\
\end{longtable}

We also printed solid end-caps to seal either end of the bellows. The
end-caps include ports for hydraulic tubing and sensors.

\hypertarget{ch1:bellows-optimization}{%
\paragraph{Bellows Optimization}\label{ch1:bellows-optimization}}

We are trying to minimize the total actuation time
(Eq.~\ref{ch1:eq:time-actuation}). This requires optimizing the maximum
bellows pressure. The bellows must also meet the output volume
requirement. The maximum pressure is limited by the feed force limit
(Eq.~\ref{ch1:eq:feed-limit}) and the squirm limit
(Eq.~\ref{ch1:eq:squirm-limit}). Both limits depend on geometry and material
properties. The feed force limit favors long skinny bellows. The squirm
limit favors short stout bellows. The output volume requires an overall
volume. For any given volume, a compromise between length and diameter
will maximize the pressure limit. The values will also depend on the number of convolutions. Adding more
convolutions can increase the pressure limit but will also limit usable
volume.

Both limits are non-linear functions of compression distance \(x\). The
feed force limit decreases with compression reaching its minimum at
full compression. The squirm limit initially decreases and then
increases with compression after a reaching a minimum. For this
optimization, the maximum available pressure is taken as the minimum of
the minimum of both limits.

\begin{figure}
\hypertarget{ch1:fig:optimization-small}{%
\centering
\includegraphics[width=\textwidth,height=\textheight,keepaspectratio]{bellows-comparison.png}
\caption{Optimization contour plot for small servo. Blue represents
maximum output volume in mL. Green represents minimum pressure limit in
kPa. The red dashed line marks the proposed design to maximize pressure for given volume. The greatest pressure limit occurs at the
intersection of the squirm pressure limit and feed force pressure
limit.}\label{ch1:fig:optimization-small}
}
\end{figure}

Convolution angle and material thickness are determined by the
manufacturing process. Elastic modulus \(E\) and Poisson's ratio \(\nu\)
are determined by material selection. Maximum servo feed force and
stroke length are determined by servo selection. The remaining three
parameters (\(D_i\), \(N\), \(h_0\)) must be selected in balance to form
a working bellows for a given output.

Using a series of contour plots to compare designs across the remaining
parameters, we can select parameters which meet our volume requirements
while maximizing output pressure. If no viable design exists, it may be
necessary to modify the number of convolutions or select a different servo.

Using the contour plots, we can design a bellows with the required
volume output, but improved pressure limits for a given servo. The
volume requirement is determined by the output actuator. From there,
maximizing the pressure limit will allow for the fastest actuation. The
pressure limit must be greater than the actuator pressure response or
the system will not work at all.
This design process is not a true pressure optimization, but can be used
to generate a viable design for further testing.
Tbl.~\ref{ch1:tbl:bellows-design} shows the proposed design and parameters.


\hypertarget{ch1:tbl:bellows-design}{}
\begin{longtable}[]{@{}
  >{\raggedright\arraybackslash}p{(\columnwidth - 4\tabcolsep) * \real{0.4810}}
  >{\raggedleft\arraybackslash}p{(\columnwidth - 4\tabcolsep) * \real{0.1519}}
  >{\raggedright\arraybackslash}p{(\columnwidth - 4\tabcolsep) * \real{0.3671}}@{}}
\caption{\label{ch1:tbl:bellows-design}Final proposed design parameters for
the hydraulic engine}\tabularnewline
\toprule\noalign{}
\begin{minipage}[b]{\linewidth}\raggedright
Parameter
\end{minipage} & \begin{minipage}[b]{\linewidth}\raggedleft
Value
\end{minipage} & \begin{minipage}[b]{\linewidth}\raggedright
Units
\end{minipage} \\
\midrule\noalign{}
\endfirsthead
\toprule\noalign{}
\begin{minipage}[b]{\linewidth}\raggedright
Parameter
\end{minipage} & \begin{minipage}[b]{\linewidth}\raggedleft
Value
\end{minipage} & \begin{minipage}[b]{\linewidth}\raggedright
Units
\end{minipage} \\
\midrule\noalign{}
\endhead
\bottomrule\noalign{}
\endlastfoot
Inner Diameter \(D_i\) & 32.2 & \si{\mm} \\
Initial Convolution Height \(h_0\) & 12.0 & \si{\mm} \\
Number of Convolutions \(N\) & 5 & - \\
Material Thickness \(t\) & 1.62 & \si{\mm} \\
Convolution Angle \(\theta\) & 90.0 & degree \\
Overall Bellows Length \(L\) & 60.0 & \si{\mm} \\
Outer Diameter \(D_o\) & 44.2 & \si{\mm} \\
Convolution Hypotenuse \(S\) & 8.5 & \si{\mm} \\
Elastic Modulus \(E\) & 12 & \si{\mega\pascal} \\
Poisson's ratio \(\nu\) & 0.26 & \\
Wall-thickness \(t\) & 1.62 & \si{\milli\meter} \\
% Eff. Area \(A_b\) & 1146084.42 & \si{\mm\squared} \\
Capacitance \(C_b\) & 0.03 & \si{\mL\per\kPa} \\
\end{longtable}

\hypertarget{ch1:zero-pressure-tests}{%
\paragraph{Zero Pressure Tests}\label{ch1:zero-pressure-tests}}

The bellows volume output and stiffness can be tested using a graduated
cylinder as an unpressurized output. 
Fig.~\ref{ch1:fig:zero-pressure-small} shows theoretical results 
for a static unpressurized bellows test.

\begin{enumerate}
\def\labelenumi{\arabic{enumi}.}
\tightlist
\item
  Set up bellows in a test stand with open output into a graduated
  cylinder
\item
  Slowly compress the bellows

  \begin{enumerate}
  \def\labelenumii{\arabic{enumii}.}
  \tightlist
  \item
    Measure output volume in the graduated cylinder.
  \item
    Measure distance compressed.
  \item
    Measure axial force using a load sensor
  \end{enumerate}
\item
  Stop at full compression

  \begin{enumerate}
  \def\labelenumii{\arabic{enumii}.}
  \tightlist
  \item
    Measure full compression.
  \end{enumerate}
\item
  Calculate axial stiffness from compression and axial force.
\end{enumerate}


\begin{figure}
\hypertarget{ch1:fig:zero-pressure-small}{%
\centering
\includegraphics[width=\textwidth,height=\textheight,keepaspectratio]{zero-pressure-test-small-bellows.png}
\caption{Example static unpressurized test for
bellows}\label{ch1:fig:zero-pressure-small}
}
\end{figure}

\hypertarget{ch1:static-pressure-tests}{%
\paragraph{Static Pressure Tests}\label{ch1:static-pressure-tests}}

The bellows capacitance and squirm limit can be tested by holding the
bellows statically and injecting known volumes of fluid into the
bellows.
Fig.~\ref{ch1:fig:static-pressure-small} shows the static pressurized test
results for bellows.

\begin{enumerate}
\def\labelenumi{\arabic{enumi}.}
\tightlist
\item
  Set up bellows in test stand with a closed output.
\item
  Compress known distance. (Starting with \SI{0}{\mm})

  \begin{enumerate}
  \def\labelenumii{\arabic{enumii}.}
  \tightlist
  \item
    Force fluid into bellows slowly until squirm failure
  \item
    Measure injected fluid volume.
  \item
    Measure internal bellows pressure.
  \item
    Measure squirm pressure as the maximum pressure reached.
  \item
    Repeat at different compression distances. Try to find compression distance
    with minimum squirm pressure.
  \end{enumerate}
\end{enumerate}

\begin{figure}
\hypertarget{ch1:fig:static-pressure-small}{%
\centering
\includegraphics[width=\textwidth,height=\textheight,keepaspectratio]{static-pressure-test-small-bellows.png}
\caption{Example static pressurized test for bellows. The blue lines
represent pressure measurements at each compression distance. The
starting internal bellows volume decreases with compression change, and
increases as fluid is injected. The slope of each blue line represents
the capacitance of the bellows at that compression distance. The green
line connects the maximum measured pressure showing the squirm limit.
The internal volume axis is flipped to represent increasing compression
distance.}\label{ch1:fig:static-pressure-small}
}
\end{figure}


The combined test results can be used to calculate bellows capacitance
and pressure limits as functions of axial compression
(Fig.~\ref{ch1:fig:results-bellows-small}).

\begin{figure}
\hypertarget{ch1:fig:results-bellows-small}{%
\centering
\includegraphics[width=\textwidth,height=\textheight,keepaspectratio]{test-results-small-bellows.png}
\caption{Example test results for bellows showing the capacitance and
pressure limits as functions of compression distance. During the initial
compression squirm limit is lower, but the drive feed limit becomes
lower with increasing compression.}\label{ch1:fig:results-bellows-small}
}
\end{figure}



\hypertarget{ch1:hydraulic-tubing}{%
\paragraph{Hydraulic Tubing}\label{ch1:hydraulic-tubing}}

The hydraulic tube transfers fluid from the drive to the output actuator
resulting in additional dynamic pressure during flow. For fast
actuation, the drive must overcome back pressure from both the output
actuator and the hydraulic tube.

We model the tube pressure differential \(p_{ba}\) is a function
(Eq.~\ref{ch1:eq:pressure-tube}) of flow rate \(\dot{V_a}\) and flow
acceleration \(\ddot{V_a}\). Flow through the tube is equal to flow into
the output actuator. The pressure response of the tube is approximated
linearly using coefficients for fluidic resistance \(R\) and fluidic
inertance \(I\). These coefficients can be fitted using empirical data
of or estimated (Eq.~\ref{ch1:eq:tube-resistance};
Eq.~\ref{ch1:eq:tube-inertance}) based on geometry.

\begin{equation}\protect\hypertarget{ch1:eq:pressure-tube}{}{ p_{ba}(\dot{V_a}, \ddot{V_a})   
\approx R \dot{V_a} + I \ddot{V_a}   
}\label{ch1:eq:pressure-tube}\end{equation}

Power \(\mathcal{P}_R\) is dissipated through the tube as a function
(Eq.~\ref{ch1:eq:tube-power}) of flowrate and pressure differential.

\begin{equation}\protect\hypertarget{ch1:eq:tube-power}{}{ \mathcal{P}_R = \dot{V_a}p_{ba}   
\approx \frac{1}{R}p_{ba}^2   
}\label{ch1:eq:tube-power}\end{equation}


\hypertarget{ch1:hydraulic-tube-design}{%
\paragraph{Hydraulic Tube Design}\label{ch1:hydraulic-tube-design}}

The hydraulic tube geometry is defined by length \(l_t\) and
inner-diameter \(d_t\). The tube resistance \(R\) depends
(Eq.~\ref{ch1:eq:tube-resistance}) on the geometry and absolute viscosity
\(\mu\) of fluid. The tube inertance \(I\) depends
(Eq.~\ref{ch1:eq:tube-inertance}) on the geometry and the density \(\rho\)
fluid. Fresh water (Tbl.~\ref{ch1:tbl:fluidic-properties}) is used
throughout all system demonstrations described in this paper, though
salt water can also be used.


\hypertarget{ch1:tbl:fluidic-properties}{}
\begin{longtable}[]{@{}lrl@{}}
\caption{\label{ch1:tbl:fluidic-properties}Approximate fluidic properties of
fresh water at room temperature}\tabularnewline
\toprule\noalign{}
Parameter & Value & Units \\
\midrule\noalign{}
\endfirsthead
\toprule\noalign{}
Parameter & Value & Units \\
\midrule\noalign{}
\endhead
\bottomrule\noalign{}
\endlastfoot
Density \(\rho\) & 1000 & \si{\kilogram\per\meter\cubed} \\
Absolute Viscosity \(\mu\) & 0.001 & \si{\pascal\second} \\
\end{longtable}

\begin{equation}\protect\hypertarget{ch1:eq:tube-resistance}{}{ R = 128 \frac{\mu l_t}{\pi d_t^4}   
}\label{ch1:eq:tube-resistance}\end{equation}
\begin{equation}\protect\hypertarget{ch1:eq:tube-inertance}{}{ I = \rho \frac{l_t}{A_T} = \frac{4 \rho}{\pi} \frac{l_t}{d_t^2}   
}\label{ch1:eq:tube-inertance}\end{equation}

If multiple tubes are used, the resistance and inertance values can be
added in series or parallel as necessary. Fittings can cause additional
resistance. Selecting low resistance fittings is an important to enable
high flow rates.

\hypertarget{ch1:hydraulic-tube-tests}{%
\paragraph{Hydraulic Tube Tests}\label{ch1:hydraulic-tube-tests}}

Resistance and inertance can be measured for the hydraulic tubes as well
as the fittings.

\begin{figure}
\hypertarget{ch1:fig:tube-test}{%
\centering
\includegraphics[width=\textwidth,height=\textheight,keepaspectratio]{tube-test.png}
\caption{Theoretical test of tube resistance and inertance per length at
different inner diameters. This analysis does not include
fittings.}\label{ch1:fig:tube-test}
}
\end{figure}

\hypertarget{ch1:system-dynamics}{%
\paragraph{System Dynamics}\label{ch1:system-dynamics}}

\begin{figure}
\hypertarget{ch1:fig:system-graph}{%
\centering
\includegraphics[width=\textwidth,height=\textheight,keepaspectratio]{system-graph.png}
\caption{System graph showing system dynamics. Axial compression
\(\dot{V}_i\) of the bellows is split between flow \(\dot{V}\) to the
actuator of capacitance \(C_a\) and radial expansion \(V_b\) of the
bellows of capacitance \(C_b\) causing back pressure
\(p_{ba} = p_b-p_a\) across the connecting tube of resistance \(R\) and
inertance \(I\) until actuator pressure \(p_a\) equalizes with bellows
pressure \(p_b\).}\label{ch1:fig:system-graph}
}
\end{figure}

Hydraulic system model (Fig.~\ref{ch1:fig:system-graph}) can be represented 
as a linear graph \cite{shearer1967introduction}. 
Servo motion is converted into axial compression. 
The displaced volume flows down the tube resulting in back-pressure
initially from dynamic effects of inertia and resistance. 
The back-pressure pressurizes the bellows through radial expansion.
Continued flow into the output actuator slowly equalizes pressure across the system.

\hypertarget{ch1:state-space-model}{%
\paragraph{State-Space Model}\label{ch1:state-space-model}}

The model only includes the hydraulic elements of the system. We assume
servo motion \(x\) is directly correlated to axial compression
\(\Delta V_{b,x}\) of the bellows and that actuator volume change
\(\Delta V_a\) is directly correlated to actuator motion. These
assumptions breakdown outside the operational limits of the engine or if
the output actuator is subject to outside forces.

By assuming a valid linear approximation for each element, we can define
a system state-space model (Eq.~\ref{ch1:eq:model-linear}) by the state
variables: hydraulic tube pressure differential \(p_{ba}\) and flowrate
\(\dot{V}_a\) through the tube into the actuator. The input variable
with is the servo velocity \(\dot{x}(t)\) as a function of time \(t\).
The system capacitance \(C\) is a result
(Eq.~\ref{ch1:eq:system-capacitance}) of the parallel capacitance of the
output actuator and the bellows.

\begin{equation}\protect\hypertarget{ch1:eq:model-linear}{}{ \left[ \begin{array}{c}   
\ddot{V}_a \\   
\dot{p}_{ba}   
\end{array}   
\right] =   
\left[\begin{array}{c}   
\frac{1}{I} p_{ba} - \frac{R}{I} \dot{V}_a \\   
-\frac{1}{C}   
 \dot{V}_a - \frac{1}{C_b} A_b \dot{x}(t)   
\end{array}\right]   
}\label{ch1:eq:model-linear}\end{equation}
\begin{equation}\protect\hypertarget{ch1:eq:system-capacitance}{}{ \frac{1}{C} = \frac{1}{C_a} + \frac{1}{C_a}   
}\label{ch1:eq:system-capacitance}\end{equation}

Given initial values for the state variables and the servo velocity
function, the state-space can be solved through integration. Then using
the state variables, we can determine the remaining system variables as
functions of time \(t\).

\[ \Delta V_{b,x}(t) = \int_0^T A_b \dot{x}(t) dt   
\] \[ \Delta V_{b,r}(t) = \Delta V_{b,x}(t) - V_a(t)   
\] \[ p_a(t) = \frac{1}{C_a} \Delta V_a(t)   
\] \[ p_b(t) = p_{ba}(t) + p_a(t)   
\]

The model assumes linear approximations for each of the elements. Model
accuracy depends on the validity of each of the linear assumptions.

\hypertarget{ch1:harmonic-oscillator-model}{%
\paragraph{Harmonic Oscillator Model}\label{ch1:harmonic-oscillator-model}}

By integrating and simplifying (Eq.~\ref{ch1:eq:harmonic-long}), the system
can be represented as a harmonic oscillator (Eq.~\ref{ch1:eq:harmonic}) with
natural frequency \(\omega_0\), damping ratio \(\zeta\), and gain \(G\).

\begin{equation}\protect\hypertarget{ch1:eq:harmonic-long}{}{ \ddot{V}    
+ \frac{R}{I} \dot{V}_a    
+ \frac{1}{I} \frac{1}{C} \Delta V_a    
= \frac{A_b}{I C_b} x(t)   
}\label{ch1:eq:harmonic-long}\end{equation}
\begin{equation}\protect\hypertarget{ch1:eq:harmonic}{}{ \ddot{V}_a   
+ 2\zeta \omega_0  \dot{V}_a^2   
+ \omega_0^2  V_a   
= G x(t)   
}\label{ch1:eq:harmonic}\end{equation}
\begin{equation}\protect\hypertarget{ch1:eq:natural-freq}{}{ \omega_0 =   
\sqrt{\frac{1}{IC}}   
}\label{ch1:eq:natural-freq}\end{equation}
\begin{equation}\protect\hypertarget{ch1:eq:damping-ratio}{}{ \zeta = \frac{R}{2}\sqrt{\frac{C}{I}}   
}\label{ch1:eq:damping-ratio}\end{equation}
\begin{equation}\protect\hypertarget{ch1:eq:input-gain}{}{ G = \frac{A_b}{I C_b}   
}\label{ch1:eq:input-gain}\end{equation}

The harmonic oscillator model can reveal how the system will react to
input motion from the servo. The servo velocity function \(\dot{x}(t)\)
can be modeled as a step function.

We used the harmonic coefficients for predictive design of the engine.
The fastest response without overshoot is achieved by designs with
critical damping (\(\zeta=1\)).

Total actuation time \(\Delta T\) is the sum Eq.~\ref{ch1:eq:time-actuation}
of servo acuation time \(\Delta T_\text{servo}\) and the system settling
time \(\Delta T_\text{settle}\). Servo actuation time is dependent
Eq.~\ref{ch1:eq:time-servo} on the servo rate and compression distance.
Settling time is defined Eq.~\ref{ch1:eq:time-settle} as four times the
reciprical of the neper frequency \(\zeta \omega_0\).

\begin{equation}\protect\hypertarget{ch1:eq:time-actuation}{}{ \Delta T = \Delta T_\text{servo} + \Delta T_\text{settle}   
}\label{ch1:eq:time-actuation}\end{equation}
\begin{equation}\protect\hypertarget{ch1:eq:time-servo}{}{ \Delta T_\text{servo} = \Delta x/v = \Delta V_{b,x}/(A\dot{x})   
}\label{ch1:eq:time-servo}\end{equation}
\begin{equation}\protect\hypertarget{ch1:eq:time-settle}{}{ \Delta T_\text{settle} := \frac{4}{\zeta \omega_0} = 8\frac{I}{R}   
}\label{ch1:eq:time-settle}\end{equation}

Minimizing actuation time is an optimization problem. For example,
settling time can be driven down by increasing resistance at the expense
of increased back pressure which may require slowing down the servo rate
increasing servo actuation time.

\hypertarget{ch1:critical-damping-is-unrealistic}{%
\paragraph{Critical Damping is
Unrealistic}\label{ch1:critical-damping-is-unrealistic}}

The tube parameters determine the settling time
(Eq.~\ref{ch1:eq:time-settle}). The settling time only depends on the tube
diameter (Eq.~\ref{ch1:eq:time-settleb}). A larger diameter tube is desired
for faster settling time and lower resistance, but large tubes are often
impractical for robotics applications. The optimal tube length can be
calculated to create critical damping.

\begin{equation}\protect\hypertarget{ch1:eq:time-settleb}{}{ \Delta T_\text{settle} = 256\frac{\mu}{\rho} \frac{1}{d_t^2}   
}\label{ch1:eq:time-settleb}\end{equation}

Critical damping where the neper frequency is equal to the natural
frequency is unrealistic. The system natural frequency is very high due
to the stiffness of the printed bellows. High bellows stiffness is
required to to avoid bellows squirm. However, as a result, critical
damping requires a long skinny tube which has very high resistance. The
high resistance increases dynamic back pressure making the system
unworkable. Therefore, we propose an under-damped system. The settling
time can still be kept low, and the settling oscillations have an
insignificant effect on output volume.

\hypertarget{ch1:control}{%
\paragraph{Control}\label{ch1:control}}

The system utilizes open-loop feed-forward control. The servo position
can be controlled electronically. Motion planning can predict the
actuator response to any input using the state-model. After servo motion
the system equalizes to known positions without need for feedback or
measurement of internal pressure. Underdamped designs may allow the
output actuator to overshoot the target geometry.

\hypertarget{ch1:system-limits}{%
\paragraph{System Limits}\label{ch1:system-limits}}

At equilibrium, dynamic pressure \(p_{ba}\) goes to zero and pressure is
equalized throughout the shared volume of the system. The equalized
actuator volume \(\Delta V_{a,\text{eq}}\) and therefore final actuator
geometry and pressure depends (Eq.~\ref{ch1:eq:equalibrium-volume}) on both
the system \(C\) and bellows \(C_b\) capacitance.

\begin{equation}\protect\hypertarget{ch1:eq:equalibrium-volume}{}{ \Delta V_{a,\text{eq}} = \frac{C}{C_b} A_b x   
}\label{ch1:eq:equalibrium-volume}\end{equation}

In most cases the bellows capacitance and therefore expansion is
significantly less than that of the actuator. In this case, system
capacitance is nearly equal to bellows capacitance resulting in a nearly
complete transfer of fluid.

The same volume output can be achieved with different bellows
geometries. However, bellows geometry also affects the pressure limits.


\hypertarget{ch1:power-energy-and-efficiency}{%
\paragraph{Power, Energy, and Efficiency
}\label{ch1:power-energy-and-efficiency}}

System work \(\mathcal{E}\) goes into actuator pressurization as well as
bellows radial pressurization, spring compression, and is dissipated by
fluidic resistance. Dissipated power and therefore efficiency \(e\) is a
function (Eq.~\ref{ch1:eq:efficiency}) of actuation rate. Additional losses
may occur through electrical and mechanical systems.

\begin{equation}\protect\hypertarget{ch1:eq:power-input}{}{ \mathcal{P}_\text{in} = F(x) \cdot \dot{x}   
}\label{ch1:eq:power-input}\end{equation}
\begin{equation}\protect\hypertarget{ch1:eq:work-in}{}{ \Delta \mathcal{E}_\text{in} = \int F(x) \cdot \dot{x} dx  
}\label{ch1:eq:work-in}\end{equation}
\begin{equation}\protect\hypertarget{ch1:eq:efficiency}{}{ e = \frac{ \Delta \mathcal{E}_\text{out} }   
            { \Delta \mathcal{E}_\text{in} }
     = \frac{ \Delta \mathcal{E}_a }   
            { \Delta \mathcal{E}_a   
            + \Delta \mathcal{E}_b   
            + \Delta \mathcal{E}_K   
            + \int \mathcal{P}_R dt }   
}\label{ch1:eq:efficiency}\end{equation}

Tbl.~\ref{ch1:tbl:final-design} shows the parameters for the proposed test
engine.

\subsection{Final Design}


\hypertarget{ch1:tbl:final-design}{}
\begin{longtable}[]{@{}
  >{\raggedright\arraybackslash}p{(\columnwidth - 4\tabcolsep) * \real{0.4810}}
  >{\raggedleft\arraybackslash}p{(\columnwidth - 4\tabcolsep) * \real{0.1519}}
  >{\raggedright\arraybackslash}p{(\columnwidth - 4\tabcolsep) * \real{0.3671}}@{}}
\caption{\label{ch1:tbl:final-design}Final proposed design parameters for
the hydraulic engine}\tabularnewline
\toprule\noalign{}
\begin{minipage}[b]{\linewidth}\raggedright
Parameter
\end{minipage} & \begin{minipage}[b]{\linewidth}\raggedleft
Value
\end{minipage} & \begin{minipage}[b]{\linewidth}\raggedright
Units
\end{minipage} \\
\midrule\noalign{}
\endfirsthead
\toprule\noalign{}
\begin{minipage}[b]{\linewidth}\raggedright
Parameter
\end{minipage} & \begin{minipage}[b]{\linewidth}\raggedleft
Value
\end{minipage} & \begin{minipage}[b]{\linewidth}\raggedright
Units
\end{minipage} \\
\midrule\noalign{}
\endhead
\bottomrule\noalign{}
\endlastfoot
\textbf{Output Actuator} (FDM Printed) & & \\
Max volume \(\Delta V_{a,\text{max}}\) & 50.0 & \si{\mL} \\
Max pressure \(p_{a,\text{max}}\) & 47.0 & \si{\kPa} \\
Capacitance \(C_a\) & 1.1 & \si{\mL\per\kPa} \\
\textbf{Linear Servo} (Festo EPCO) & & \\
Feed Force \(F_\text{max}\) & 105.0 & \si{\newton} \\
Stroke \(x_\text{max}\) & 100.0 & \si{\mm} \\
Feed Velocity \(v_\text{max}\) & 500.0 & \si{\mm\per\second} \\
Feed Acceleration \(a_\text{max}\) & 10000.0 &
\si{\mm\per\second\squared} \\
\textbf{Bellows} (FDM Printed) & & \\
Inner Diameter \(D_i\) & 32.2 & \si{\mm} \\
Initial Convolution Height \(h_0\) & 12.0 & \si{\mm} \\
Number of Convolutions \(N\) & 5 & - \\
Material Thickness \(t\) & 1.62 & \si{\mm} \\
Convolution Angle \(\theta\) & 90.0 & degree \\
Overall Bellows Length \(L\) & 60.0 & \si{\mm} \\
Outer Diameter \(D_o\) & 44.2 & \si{\mm} \\
Convolution Hypotenuse \(S\) & 8.5 & \si{\mm} \\
Elastic Modulus \(E\) & 12 & \si{\mega\pascal} \\
Poisson's ratio \(\nu\) & 0.26 & \\
Wall-thickness \(t\) & 1.62 & \si{\milli\meter} \\
% Eff. Area \(A_b\) & 1146084.42 & \si{\mm\squared} \\
Capacitance \(C_b\) & 0.03 & \si{\mL\per\kPa} \\
\textbf{Hydraulic Tube} (Plastic) & & \\
Inner Diameter \(d_t\) & 4.0 & \si{\mm} \\
Length \(l_t\) & 2.0 & \si{\m} \\
\textbf{System} & & \\
Capacitance \(C\) & 0.03 & \si{\mL\per\kPa} \\
Resistance \(R\) & 0.32 & \si{\kPa\per\mL\s} \\
Inertance \(I\) & 0.16 & \si{\kPa\per\mL\s\squared} \\
Nat Freq \(\omega_0\) & 2.20 & \si{\hertz} \\
Damp. Rat. \(\zeta\) & 0.07 & - \\
Settling Time \(\Delta T_t\) & 4.0 & \si{\s} \\
\end{longtable}


\hypertarget{ch1:dynamic-tests}{%
\paragraph{Dynamic Tests}\label{ch1:dynamic-tests}}

Finally, the engine dynamics can be tested using a known actuator.

\begin{enumerate}
\def\labelenumi{\arabic{enumi}.}
\tightlist
\item
  Set up bellows in test stand w/ hydraulic tube to known actuator
\item
  Drive bellows at known velocity/acceleration to full compression

  \begin{enumerate}
  \def\labelenumii{\arabic{enumii}.}
  \tightlist
  \item
    Measure compression distance.
  \item
    Measure bellows internal pressure.
  \item
    Measure actuator internal pressure.
  \item
    Measure axial force between the servo and bellows .
  \item
    Repeat for at different velocity/acceleration
  \end{enumerate}
\item
  Repeat for different actuators and engine designs.
\end{enumerate}

\begin{figure}%[h]
\hypertarget{ch1:fig:dynamics-small}{%
\centering
\includegraphics[width=\textwidth,height=\textheight,keepaspectratio]{labtest.png}
\caption{(LEFT) Theoretical dynamic test results for the proposed engine
design. (RIGHT) Measured results of a dynamic test using the proposed engine.}\label{ch1:fig:dynamics-small}
}
\end{figure}


Fig.~\ref{ch1:fig:dynamics-small} shows simulated and measured test results for the proposed engine. The system is very under-damped. However, the proposed
design remains under the pressure limits and actuates in \SI{5}{\second}
with a \SI{4}{\second} settling time.

During the settling time, large fluctuations occur in the bellows
internal pressure. However, the output actuator volume shows much less
significant fluctuations.

\hypertarget{ch1:conclusion}{%
\section{Conclusion}\label{ch1:conclusion}}

\hypertarget{ch1:success}{%
\paragraph{Success}\label{ch1:success}}

The hydraulic drive deployed for multiple dives on two separate
platforms and controlled a variety of fluidic actuators in lab and sea
tests. In comparison to previous deep sea hydraulic drives, this design
performs faster actuation, while using less power in more compact
package. The drive was deployed on large \gls{ROV} platforms, but its size and
power requirements will allow deployment on much smaller platforms.

\hypertarget{ch1:conclusions}{%
\paragraph{Conclusions}\label{ch1:conclusions}}

Tests showed important limitations to the system which can be addressed
in future research. First, The high flow rates generated by the drive
were limited by back pressure from viscous and inertial effects, which
be seen in the delayed response and pressure overshoot seen in the lab
test results in Fig.~\ref{ch1:fig:dynamics-small} The length and diameter of
the hydraulic tube is a major factor. Shorter or wider tubes would
transmit hydraulic flow with less dynamic losses. The dynamic affects
also make the drive and actuator a second order hydraulic system.

Second, the elastic bellows proved not ideal for converting linear
motion to flow. Elastic buckling limited the length to diameter ratio of
the bellows. The higher diameter bellows required higher forces and
larger electric actuators especially given additional back pressure from
the dynamic flow affects. Any modifications and guides designed to limit
buckling added significant friction or undesired complexity to the
system. Further investigation is required into bellows design or
alternatives.

At this point, the test results shown are purely theoretical. The
results are not very promising and suggest a larger engine is required
to drive actuators of the size. A larger engine would allow for a
greater feed force limit allowing for wider bellows with higher squirm
limits. Increasing the overall pressure limit would allow the drive to
operate faster and overcome the resulting dynamic pressure.

Finally, the drive is designed to efficiently and precisely control a
single degree of freedom actuator set. An array of drives could allow
for control of complex actuator sets while serving on relatively compact
and low power platforms.

\hypertarget{ch1:future-work}{%
\paragraph{Future Work}\label{ch1:future-work}}

A system prototype was demonstrated in an operational environment.
However, the design process still requires further laboratory data to be
considered validated. Additional field-testing should be conducted on
SWAP-C platforms to further demonstrate effectiveness in the operational
environment.

We identified the bellows pressure limits as the primary limiting factor
for actuation rate. Future work should be done to investigate potential
comparisons between bellows, traditional pistons, and alternative
solutions such as \gls{LSRD} \cite{gruebele2019long}. Additionally, effort should be made to compare the single stroke linear pump design to rotary pumps.